\documentclass[12pt,a4paper]{article}
\usepackage[T1]{fontenc}
\usepackage{graphicx}
\usepackage{mathtools}
\usepackage{amssymb}
\usepackage{amsthm}
\usepackage{nameref}
\usepackage{hyperref}
\usepackage{listings}
\usepackage{hyperref}

\usepackage[left=2cm,right=2cm,top=2cm,bottom=2cm]{geometry}
\title{About the computation of the free electrochemical energy with \texttt{ei-compute-fee}.}
\author{Pierre Beaujean}
\begin{document}
	\maketitle
	
	\begin{abstract}
		The program \texttt{ei-compute-fee} from the \texttt{ec-interface} package,\footnote{\url{https://github.com/pierre-24/ec-interface/}} can be used to compute the free electrochemical energy of a system ($\Omega$) following the works of Prof. J.S. Filhol. 
	\end{abstract}
	
	In theory, the free electrochemical energy (or grand potential) of a system is given by:\begin{equation}
		\Omega(\Phi) = H(\Phi) - q\Phi,\label{eq:gdp}
	\end{equation}
	where $q$ is the surface charge at a given electrode potential $\Phi$ and $H(\Phi)$ is the enthalpy of the system. However, introducing or removing a certain amount of electron to the system, $N_e = N - N_0$ (where $N$ is the number of electrons used in the calculation and $N_0$ is the number for the neutral system), provide a way to estimate $\Omega$. According to Hagopian et al. (see Eq.~12 and 13 of \cite{hagopianAdvancementHomogeneousBackground2022}), a good approximation is given by:\begin{equation}
		\Omega(\Phi) = A_{\text{VASP}}(N_e) - N_e\,E_F(N_e),\label{eq:gdp-fermi}
	\end{equation} 
	where $A_{\text{VASP}}$ and $E_F$ are the free (\texttt{E0} in \texttt{OSZICAR}) and Fermi energies reported by VASP for a given $N_e$, respectively. However, the ``true'' work function $\Phi$ is better approximated by:\begin{equation}
		\Phi(N_e) = E_V(N_e)-E_F(N_e),\label{eq:phi}
	\end{equation} 
	where $E_V$ is the vacuum potential (evaluated from \texttt{LOCPOT} at the location of the center of the vacuum between the electrodes). These formulas are used to compute the values of $\Omega$ reported by default (or using \texttt{\lstinline|--hbm-fermi|}).
	
	\paragraph{With the PBM approach.} Within the Poisson-Boltzmann (PB) method, Eq.~\eqref{eq:gdp} becomes:\begin{equation}
		\Omega(\Phi) = A_{\text{VASP}}(N_e) + N_e\,\Phi(N_e). \label{eq:gdp-pbm}
	\end{equation}
	This formula is used to compute the values of $\Omega$ reported using  \texttt{\lstinline|--pbm|}.
	
	\paragraph{With the HBM approach.} According to Kopa{\v c} Lautar et al. \cite{kopaclautarModelingInterfacialElectrochemistry2020}, given a fraction of active electrode $\alpha_s$ so that $N_s = \alpha_s\,N_e$, one gets:\begin{equation}
		\Omega(\Phi) =  A_{\text{VASP}}^0 + \alpha_s\,\left\{A_{\text{VASP}}(N_e)-A_{\text{VASP}}^0+ N_e\,\Phi(N_e)-\int_0^{N_e} E_V(n)\,dn\right\}, \label{eq:gdp-hbm}
	\end{equation}
	where $A_{\text{VASP}}^0$ is the free energy reported by VASP without any excess electron. The integral accounts for the homogeneous background (HB) of compensating charges.\footnote{And the evolution of $E_V$ with $n$ is not linear, so this integral does not simply evaluate to $E_V(N_e)$!} This formula is used to compute the values of $\Omega$ reported using  \texttt{\lstinline|--hbm| $\alpha_s$}.
	
	
	One way to evaluate $\alpha_s$ is:\begin{equation*}
		\alpha_s^c = \frac{d_0}{d},
	\end{equation*}
	where $d_0$ and $d$ are the vacuum thickness and the size of the unit cell, respectively. However, it is also possible to extract this quantity from a ratio of capacitances evaluated with two different techniques: \cite{hagopianAdvancementHomogeneousBackground2022}\begin{equation*}
		\alpha_s = \frac{C_{\Omega\Phi}}{C_{\Phi N}}, \text{ where }  \frac{1}{C_{\Phi N}} = \left.-\frac{1}{e}\frac{\partial\Phi}{\partial N_e}\right|_{N_e=0} \text{ and } C_{\Omega\Phi} = \left.-\frac{\partial^2\Omega}{\partial\Phi^2}\right|_{N_e=0},
	\end{equation*}
	where $\Phi$ is evaluated using Eq.~\eqref{eq:phi}, but $\Omega$ is approximated using Eq.\eqref{eq:gdp-fermi}. When using \texttt{\lstinline|--hbm-ideal|}, this formula is used to evaluate $\alpha_s$, and then Eq.~\eqref{eq:gdp-hbm} is used to compute the values of $\Omega$.
	
	\bibliographystyle{unsrt}
	\bibliography{potential}
\end{document}