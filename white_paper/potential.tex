\documentclass[12pt,a4paper]{article}
\usepackage[T1]{fontenc}
\usepackage{graphicx}
\usepackage{mathtools}
\usepackage{amssymb}
\usepackage{amsthm}
\usepackage{nameref}
\usepackage{hyperref}
\usepackage[left=2cm,right=2cm,top=2cm,bottom=2cm]{geometry}
\title{About the computation of the free electrochemical energy}
\author{Pierre Beaujean}
\begin{document}
	\maketitle
	
	\paragraph{TL;DR:} The \texttt{ec-interface} package reports $\Omega = A_{\text{VASP}} - N_e\,E_F$, which is \textbf{an approximation} of the correct free electrochemical energy! Depending on the case, you need either Eq.~\eqref{eq:gdp-pbm} or \eqref{eq:gdp-hbm}.\\
	
	The free electrochemical energy of a system is given by:\begin{equation}
		\Omega(\Phi) = H(\Phi) - q\Phi,\label{eq:gdp}
	\end{equation}
	where $q$ is the surface charge at a given electrode potential $\Phi$ and $H(\Phi)$ is the enthalpy of the system.
	
	\paragraph{With the PBM approach.} According to Hagopian et al. (see Eq.~12 of \cite{hagopianAdvancementHomogeneousBackground2022}), one can introduce $N_e$ as the number of excess electron in the system and define\begin{equation}
		\Phi(N_e) = E_V(N_e)-E_F(N_e)
	\end{equation} 
	where $E_F$ is the Fermi energy and $E_V$ is the vacuum potential (evaluated from \texttt{LOCPOT} in the vacuum between the electrodes). Within the Poisson-Boltzmann (PB) method, Eq.~\eqref{eq:gdp} becomes:\begin{equation}
		\Omega(\Phi) = A_{\text{VASP}}(N_e) + N_e\,\Phi(N_e), \label{eq:gdp-pbm}
	\end{equation}
	where $A_{\text{VASP}}$ is the free energy reported by VASP (\texttt{E0} in \texttt{OSZICAR}) for a given $N_e$.
	
	\paragraph{With the HBM approach.} According to Kopa{\v c} Lautar et al. \cite{kopaclautarModelingInterfacialElectrochemistry2020}, given a vacuum fraction $\alpha_s$ so that $N_s = \alpha_s\,N_e$, one gets:\begin{equation}
		\Omega(\Phi) =  A_{\text{VASP}}^0 + \alpha_s\,\left\{A_{\text{VASP}}(N_e)-A_{\text{VASP}}^0+ N_e\,\Phi(N_e)-\int_0^{N_e} E_V(n)\,dn\right\}, \label{eq:gdp-hbm}
	\end{equation}
	where $A_{\text{VASP}}^0$ is the free energy reported by VASP without any excess electron. The integral accounts for the homogeneous background (HB) of compensating charges.\footnote{And the evolution of $E_V$ with $n$ is not linear, so this integral does not simply evaluate to $E_V(N_e)$!} One way to evaluate $\alpha_s$ is:\begin{equation*}
		\alpha_s^c = \frac{d_0}{d},
	\end{equation*}
	where $d_0$ and $d$ are the vacuum thickness and the size of the unit cell, respectively. However, it is also possible to extract this quantity from a ratio of capacitances evaluated with two different techniques: \cite{hagopianAdvancementHomogeneousBackground2022}\begin{equation*}
		\alpha_s = \frac{C_{\Omega\Phi}}{C_{\Phi N}}, \text{ where }  \frac{1}{C_{\Phi N}} = \left.-\frac{1}{e}\frac{\partial\Phi}{\partial N_e}\right|_{N_e=0} \text{ and } C_{\Omega\Phi} = \left.-\frac{\partial^2\Omega}{\partial\Phi^2}\right|_{N_e=0}.
	\end{equation*}
	
	\bibliographystyle{unsrt}
	\bibliography{potential}
\end{document}